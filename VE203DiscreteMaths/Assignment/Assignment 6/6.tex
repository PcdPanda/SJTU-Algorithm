\documentclass[12pt]{article}
\usepackage{amsmath}
\usepackage{amssymb}
\usepackage{geometry}
\usepackage{enumerate}
\usepackage{natbib}
\usepackage{float}%稳定图片位置
\usepackage{graphicx}%画图
\usepackage[english]{babel}
\usepackage{a4wide}
\usepackage{indentfirst}%缩进
\usepackage{enumerate}%加序号
\usepackage{multirow}%合并行
\title{\large UM-SJTU JOINT INSTITUTE\\DISCRETE MATHEMATICS\\(VE203)\\\ \\\ \\\ \\\ \\\ \\\ \\\ \\\ \\\ \\\ \\\
ASSIGNMENT 6\\\ \\\ \\\ \\\ \\\ \\\ }
\author{Name: Pan Chongdan\\ID: 516370910121}
\date{Date: \today}


\begin{document}
\maketitle
\newpage
\section{Q1}
$p$ is prime $\Rightarrow \varphi(p)=p-1$,so when $k=1,\varphi(p^k)=p^k-p^{k-1}$
\\Assume when $k=n,\varphi(p^k)=p^k-p^{k-1}$
\\Then when $k=n+1$, it's clear that numbers are relative prime to $p^k$ are still relative to $p^{k+1}$. If $a$ and $p^k$ are relative prime, then gcd($a,p^{k+1}$)=1$\Rightarrow$gcd$(a+n\cdot p^k,p^{k+1})=1(n<p)$
\\$\therefore\forall a$,there will exist more (p-1) numbers which relative prime to $p^(k+1)$
\\$\therefore,\varphi(p^{k+1})=p\cdot\varphi(p^k)=p^{k+1}-p^k$
\\According to induction,$\varphi(p^k)=p^k-p^{k+1}$
\section{Q2}
$n^3+2n=n(n^2+2)$
\par$n(-n^3-3n)+n^4+3n^2+1=1\Rightarrow n$ and $n^4+3n^2+1$ are relatively prime.
\par$(n^2+2)(n^2+1)-(n^4+3n^2+1)=1\Rightarrow n^2+2$ and $n^4+3n^2+1$ are relatively prime.
\par$\therefore n^3+2n$ and $n^4+3n^2+1$ are relatively prime.
\section{Q3}
Assume $G=\{a^n|n\in\mathbb{Z}\},H\le G,H=\{a^k|k\in\mathbb{Z}\}$,if $i$ is the least number of $k$. According to the division algorithm, $\forall k=mi+j(j<i,m\in\mathbb{Z})$
\par If $j=0\Rightarrow i=mn\Rightarrow n\mid i\Rightarrow H$ is cyclic.
\par If $\exists j\neq0$, which means $H$ is not cyclic $\Rightarrow a^{-k}=a^{-mi-j}$
\par $\because a^i\in H\Rightarrow a^{mi}\in H\Rightarrow a^{mi}\cdot a^{-mi-j}=a^{-j}\in H\Rightarrow a^j\in H\Rightarrow j<i$, which leads a contradiction $i$ is the least number of $k$.
\par $\therefore j=0$ and $H$ is cyclic.
\section{Q4}
Assume $3\nmid ab\Rightarrow (3\nmid a)\wedge(3\nmid b)$
\par if $a=3k+1,(k\in\mathbb{Z})\Rightarrow a^2=9k^2+6k+1\Rightarrow a^2\equiv1$ (mod 3)
\par if $a=3k+2,(k\in\mathbb{Z})\Rightarrow a^2=9k^2+12k+4\Rightarrow a^2\equiv1$ (mod 3)
\par$\therefore a^2\equiv b^2\equiv1$ (mod 3)$\Rightarrow c^2\equiv a^2+b^2\equiv 2$ (mod 3)
\par if $c=3k,c^2=9k^2\equiv0$ (mod 3) 
So there doesn't exist $c$ such $c^2\equiv2$ (mod 3), which leads to a contradiction.
\par$\therefore3\mid ab$
\section{Q5}
$((\mathbb{Z}/11\mathbb{Z})^*,\otimes_{11})=\{[1]_{11},[2]_{11},[3]_{11},[4]_{11},[5]_{11},[6]_{11},[7]_{11},[8]_{11},[9]_{11},[10]_{11}\}$
\par $[2]_{11}^2=[4]_{11},[2]_{11}^3=[8]_{11},[2]_{11}^4=[5]_{11},[2]_{11}^5=[10]_{11},[2]_{11}^6=[9]_{11},[2]_{11}^7=[7]_{11}$
\par$[2]_{11}^8=[3]_{11},[2]_{11}^2=[6]_{11},[2]_{11}^10=[1]_{11}$
\par The generator is $[2]_{11}$
\section{Q6}
$e=[1]_{89},[12]_{89}\otimes[52]_{89}=[1]_{89}$
\par The inverse is $[52]_{89}$
\section{Q7}
$[27]_{56}^2=[1]_{56}=e$
\par Its order is 2.
\section{Q8}
$((\mathbb{Z}/14\mathbb{Z}^*,\otimes_{14})=\{[1]_{14},[3]_{14},[5]_{14},[9]_{14},[11]_{14},[13]_{14}\}$
\par $[3]_{14}^2=[9]_{14},[3]_{14}^3=[13]_{14},[3]_{14}^4=[11]_{14},[3]_{14}^5=[5]_{14},[3]_{14}^6=[1]_{14}$
\par $\therefore((\mathbb{Z}/14\mathbb{Z}^*$ is a cyclic group.
\section{Q9}
\begin{table}[H]
\centering

\begin{tabular}{|c|c|c|c|c|c|c|}
\hline
$\otimes9$&$[1]_9$  &$[2]_9$  &$[4]_9$  &$[5]_9$  &$[7]_9$  &$[8]_9$  \\ \hline
$[1]_9$ &$[1]_9$  &$[2]_9$  &$[4]_9$  &$[5]_9$  &$[7]_9$  &$[8]_9$  \\ \hline
$[2]_9$ &$[2]_9$  &$[4]_9$  &$[8]_9$  &$[1]_9$  &$[5]_9$  &$[7]_9$  \\ \hline
$[4]_9$ &$[4]_9$  &$[8]_9$  &$[7]_9$  &$[2]_9$  &$[1]_9$  &$[5]_9$  \\ \hline
$[5]_9$ &$[5]_9$  &$[1]_9$  &$[2]_9$  &$[7]_9$  &$[8]_9$  &$[4]_9$  \\ \hline
$[7]_9$ &$[7]_9$  &$[5]_9$  &$[1]_9$  &$[8]_9$  &$[4]_9$  &$[2]_9$  \\ \hline
$[8]_9$ &$[8]_9$  &$[7]_9$  &$[5]_9$  &$[4]_9$  &$[2]_9$  &$[1]_9$  \\ \hline
\end{tabular}
\caption{Cayley Table}
\end{table}
It's cyclic because $[2]_9^2=[4]_9,[2]_9^3=[8]_9,[2]_9^4=[7]_9,[2]_9^5=[5]_9,[2]_9^6=[1]_9$ 
\end{document}