\documentclass[12pt]{article}
\usepackage{amsmath}
\usepackage{amssymb}
\usepackage{geometry}
\usepackage{enumerate}
\usepackage{natbib}
\usepackage{float}%稳定图片位置
\usepackage{graphicx}%画图
\usepackage[english]{babel}
\usepackage{a4wide}
\usepackage{indentfirst}%缩进
\usepackage{enumerate}%加序号
\usepackage{multirow}%合并行
\title{\large UM-SJTU JOINT INSTITUTE\\DISCRETE MATHEMATICS\\(VE203)\\\ \\\ \\\ \\\ \\\ \\\ \\\ \\\ \\\ \\\ \\\
ASSIGNMENT 7\\\ \\\ \\\ \\\ \\\ \\\ }
\author{Name: Pan Chongdan\\ID: 516370910121}
\date{Date: \today}


\begin{document}
\maketitle
\newpage
\section{Q1}
Assume A only contains finite $n$ prime numbers such as $p(1)=3,p(2)=7\cdots p(n)$. Then assume $Q=4\cdot p(1)\cdot p(2)\cdots p(n)+3\in A$. It's clear that $Q$ is odd and its divisor can be $4k+1$ because $(4k_1+1)\cdot(4k_2+1)=4(4k_1k_2+k_1+k_2)+1\notin A$ So the only divisor of $Q$ is 1 and itself, which means $Q$ is another prime, so A contains infinity primes.
\section{Q2}
Assume $g=gcd(c,m),m=xg,c=yg\Rightarrow\frac{m}{g}=x$
\\$ac\equiv bc\mod m\Rightarrow ayg=byg+kxg\Rightarrow a\equiv b+\frac{k}{y}x$
\\$a-b\in\mathbb{Z}\Rightarrow\frac{kx}{y}\in\mathbb{Z}$
\\If $\frac{k}{y}\in\mathbb{Z}\Rightarrow a\equiv b\mod x$, else $\frac{x}{y}\in\mathbb{Z}\Rightarrow g\neq gcd(c,m)$
\\$\therefore a\equiv b\mod\frac{m}{gcd(c,m)}$
\section{Q3}
We can solve the equation by solve
$$75=36x+1309y$$
$$1309=36\cdot36+13$$
$$36=2\cdot13+10$$
$$13=10+3$$
$$10=3\cdot3+1$$
$$1=-11\cdot1309+400\cdot36$$
$$75=-825\cdot139+30000\cdot36$$
$$\therefore x=30000$$
\section{Q4}
$$2x\equiv4\mod5\Rightarrow6x\equiv12\mod15$$
$$3x\equiv5\mod7\Rightarrow6x\equiv10\mod14$$
$$\therefore6x\equiv192\mod210\Rightarrow 7x\equiv224\mod245$$
$$\because 7x\equiv2\mod13\Rightarrow 7x\equiv1939\mod3185$$
$$\therefore x\equiv 277\mod455$$
\section{Q5}
$$x\equiv5\mod6\wedge x\equiv8\mod15\Rightarrow x\equiv23\mod30$$
$$\because x\equiv3\mod10\Rightarrow x\equiv23\mod30$$
$$\therefore x=23\mod30$$
\section{Q6}
$$x\equiv5\mod 5\wedge x\equiv3\mod7\Rightarrow x\equiv10\mod35$$
$$x\equiv8\mod 11\wedge x\equiv2\mod17\Rightarrow x\equiv19\mod187$$
$$x\equiv1515\mod 6545$$
\section{Q7}
Assume $f=n\log_2(n),g=\log_2(n),f$ is $O(g)\Rightarrow |n\log_2(n)|\leq C|\log_2(n)|$
\\if $n\neq 1\Rightarrow |n|\leq C\Rightarrow C$ doesn't exist since $n$ can be any number.
\section{Q8}
They're all order $\log_2{n}$, it's clear that $|\log_2{n}|\leq|\log_2{n}|$
$$|\log_{10}{n}|=|\frac{\log_2{n}}{\log_{2}{10}}|\leq|\log_2{n}|$$
$$|\log_2n|=|\frac{\log_{10}n}{\log_{10}2}|\leq4|\log_{10}n|$$
\\$\therefore \log_{10}n$ is order $\log_2n$, similarly:
$$|\ln{n}|=|\frac{\log_2{n}}{\log_2{e}}|\leq|\log_2{n}|$$
$$|\log_2n|=|\frac{\ln n}{\ln2}|\leq2|\ln n|$$
\\$\therefore$ they have the same order $\log_2{n}$
\section{Q9}
\begin{enumerate}[(i)]
\item 
$$\frac{1}{n}=\int_{n-1}^n\frac{1}{n}\mathrm{d}x\Rightarrow\frac{1}{n}<\int_{n-1}^n\frac{1}{x}\mathrm{d}x\Rightarrow\sum_{j=2}^n\frac{1}{j}<\int_{2-1}^2\frac{1}{x}\mathrm{d}x+\cdots\int_{n-1}^n\frac{1}{x}\mathrm{d}x=\int_1^n\frac{1}{x}\mathrm{d}x$$ 
\item 
$$H(n)=\frac{1}{n}+\frac{1}{n-1}\cdots+\frac{1}{1}=1+\sum_{j=2}^n\frac{1}{j}<1+\int_1^n\frac{1}{x}\mathrm{d}x=1+\ln(n)$$
$$\therefore\lim_{n\to\infty}|\frac{H(n)}{\ln(n)}|<\lim_{n\to\infty}|\frac{1}{\ln(n)}+1|=1$$
$$\frac{1}{n}=\int_n^{n+1}\frac{1}{n}\mathrm{d}x\Rightarrow\frac{1}{n}>\int_n^{n+1}\frac{1}{x}\mathrm{d}x\Rightarrow H(n)>\int_1^n\frac{1}{x}\mathrm{d}x=\ln(n)\Rightarrow\lim_{n\to\infty}|\frac{H(n)}{\ln(n)}|>1$$
$$\therefore\lim_{n\to\infty}|\frac{H(n)}{\ln(n)}|=1$$
$\therefore H(n)$ is $O(\ln(n))$
\end{enumerate}
\section{Q10}
$$\lfloor x^3-4\rfloor=x^3-y-4(0\leq y<1)$$
$$\lim_{x\to\infty}\frac{\lfloor x^3-4\rfloor}{|x^3|}=\lim_{x\to\infty}1+|\frac{y+4}{x^3}|=1(0\leq y+4<5)$$
$\therefore\lfloor x^3-4\rfloor$ is $O(x^3)$
$$\lim_{x\to\infty}\frac{|x^3|}{\lfloor x^3-4\rfloor}=\lim_{x\to\infty}=|\frac{x^3}{x^3-y-4}|=|1+\frac{y+4}{x^3-y-4}|=|1|$$
$\therefore\lfloor x^3-4\rfloor$ is order $x^3$
\section{Q11}
$$\lim_{n\to\infty}\frac{|n^{n-k}|}{|n^n|}=\lim_{n\to\infty}\frac{1}{|n^k|}=0$$
$\therefore n^{n-k}$ is $O(n^n),$ if $n^n$ is $O(n^{n-k})$ then there exists $C$ such that $\forall n,|n^n|\leq C|n^{n-k}|$
$$\therefore \forall n\in\mathrm{dom},k>1,C\geq|n^k|$$
$\therefore$ if the two functions' domain has both upper bound $a$ and lower bound $b$ than $C\geq\max(|a^k|,|b^k|)$, otherwise, such as the domain is $\mathbb{R}$ or $\mathbb{N}$ they don't have the same order.
\end{document}