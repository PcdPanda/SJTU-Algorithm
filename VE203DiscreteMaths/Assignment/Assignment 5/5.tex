\documentclass[12pt]{article}
\usepackage{amsmath}
\usepackage{amssymb}
\usepackage{geometry}
\usepackage{enumerate}
\usepackage{natbib}
\usepackage{float}%稳定图片位置
\usepackage{graphicx}%画图
\usepackage[english]{babel}
\usepackage{a4wide}
\usepackage{indentfirst}%缩进
\usepackage{enumerate}%加序号
\usepackage{multirow}%合并行
\title{\large UM-SJTU JOINT INSTITUTE\\DISCRETE MATHEMATICS\\(VE203)\\\ \\\ \\\ \\\ \\\ \\\ \\\ \\\ \\\ \\\ \\\
ASSIGNMENT 5\\\ \\\ \\\ \\\ \\\ \\\ }
\author{Name: Pan Chongdan\\ID: 516370910121}
\date{Date: \today}


\begin{document}
\maketitle
\newpage
\section{Q1}
\begin{enumerate}[(i)]
\item $e\cdot x^{-1}\in H\Rightarrow x^{-1}\in H$
\item $y^{-1}\in H\Rightarrow x\cdot (y^{-1})^{-1}\in H\Rightarrow x\cdot y\in H$
\end{enumerate}
\section{Q2}
\begin{enumerate}
\item\{e\}\item\{e,(01)(23)\}\item\{e,(02)(13)\}\item\{e,(13)\}\item\{e,(30)(12)\}\item\{e,(20)\}
\item \{e,(0123),(0321)\}\item\{e,(01)(23),(02)(13),(13),(30)(12),(20),(0123),(0321)\}
\item \{e,(01)(23),(02)(13),(30)(12))\} \item\{e,(20)(13),(0123),(0321)\}
\end{enumerate}
\section{Q3}
$G$'s order is 10 $\Rightarrow \forall g\in G, g^{10}=e\Rightarrow (g^5)^2=e $
\par $\because g^5\in G\therefore g^5$'s order is 2.
\section{Q4}
$(G,\cdot)=S_n$,where n is infinite
$\forall m,(012\cdots m)^m=e$
\section{Q5}
$\textbf{(0123)}^{-1}=\textbf{(3210)},\textbf{(01)}(0123)=\textbf{(123)},(01)(3210)=\textbf{(203)},(203)(123)=\textbf{(310)}$
\\ $(123)(203)=\textbf{(012)},(123)^{-1}=\textbf{(132)},(203)^{-1}=\textbf{(023)},(310)^{-1}=\textbf{(301)},(012)^{-1}=\textbf{(021)}$
\\ $(01)(310)=\textbf{(30)},(01)(301)=\textbf{(31)},(01)(012)=\textbf{(12}),(01)(021)=\textbf{(02)}$ 
\\ $(02)(310)=\textbf{(3120)},(3120)^{-1}=\textbf{(0213)},(02)(301)=\textbf{(3201)},(3201)^{-1}=\textbf{(1023)},(02)(023)=\textbf{(23)},\textbf{(01)(23)},\textbf{(02)(13)},\textbf{(03)(12)},\textbf{e}$
\par So there are total 24=4! elements in <(0123),(01)> and it's equal to $S_4$
\section{Q6}
Assume $a=(01)(02)=\textbf{(021)},b=\textbf{(01)(23)},a^{-1}=\textbf{(012)},a\cdot b=\textbf{(123)}$
\\$(123)^{-1}=\textbf{(321)},(01)(23)=\textbf{(013)},(013)^{-1}=\textbf{(031)}$
\\$(01)(23)(031)=\textbf{(023)},(023)^{-1}=\textbf{(032)},(012)(013)=\textbf{(13)(02)},(012)(032)=\textbf{(03)(12)},\textbf{e}$
\\There are total 12 elements for $A_4$, so there exists
\section{Q7}
\begin{enumerate}
\item 
\par Assume $ghg^{-1}\notin H,a\in H,(ghg^{-1})a=b\in H$
\\Then $(ghg^{-1})aa^{-1}=ba^{-1}=ghg^{-1}$
\\$\because ba^{-1}\in H\therefore$ there is a contradiction for $ghg^{-1}\notin H$ if $(ghg^{-1})a\in H$
\\$\therefore (ghg^{-1})a\notin H\Rightarrow (ghg^{-1})H\neq H$
\\$\because (ghg^{-1})=(gH)\star(hH)\star(g^{-1}H)=(gH)\star(g^{-1}H)=eH=H$
\\So $\star$ is invalid if $ghg^{-1}\notin H$ and $H$ must be normal
\\If $H$ is normal and $\star$ is valid:
\\$[(aH)\star(bH)]\star(cH)=(a\cdot b)H\star(cH)=[(a\cdot b)\cdot c]H$
\\$(aH)\star[(bH)\star(cH)]=aH\star[(b\cdot c]H)=[a\cdot(b\cdot c)]H=[(a\cdot b)\cdot c]H$
\\$\therefore [(aH)\star(bH)]\star(cH)=(aH)\star[(bH)\star(cH)]$
\\$\star$ is associative of $X$
\\$(aH)\star(eH)=(a\cdot e)H=aH$
\\$eH$ is the identify of $X$
\item Assume $G=S_3,H=S_2,g=(02),h=(01)\Rightarrow ghg^{-1}=(02)(01)(02)=(21)\notin H$
\\$(02)H=(012)H=\{(02),(012))\},(21)H=(021)H=\{(12),(021)\}$\\
\\According to the definition of $\star, (02)H\star (21)H=(021)H,(012)H\star(021)H=H$, which is a contradiction and $\star$ is not a function
\\So $(X,\star)$ is not a group.
\end{enumerate}
\section{Q8}
\begin{enumerate}[(i)]
\item When n=1,it's clear that $ab=ab$,assume $(ab)^m=a^mb^m$
\\when n=m+1,$(ab)^{m+1}=a^mb^mab=a^mb^{m-1}ab^2=\cdots =a^{m+1}b^{m+1}$
\\$\therefore (ab)^n=a^nb^n$
\item $\forall a,b\in H$, assume $a,b$'s order are $m,n.$
\\$(ab)^{mn}=a^{mn}b^{mn}=e^(n+m)=e\Rightarrow ab\in H,(ab)^{-1}=(ab)^{mn-1}$.
\\$\because (ab)^{mn-1}=a^{mn-1}b^{mn-1}\therefore (ab)^{mn-1}\in H$
\\$\therefore H$ is a group and $H\leq G$
\end{enumerate}
\section{Q9}
\begin{enumerate}
\item $$e=\begin{pmatrix}
1&0\\0&1
\end{pmatrix}$$
$$A^2=\begin{pmatrix}
-1&-1\\1&0
\end{pmatrix}$$
$$A^3=\begin{pmatrix}
1&0\\0&1
\end{pmatrix}=e$$
A's order is 3
\item 
$$B^2=\begin{pmatrix}
-1&0\\0&-1
\end{pmatrix}$$
$$B^4=\begin{pmatrix}
1&0\\0&1
\end{pmatrix}=e$$
B's order is 4
\item 
Assume 
$$C=A\cdot B=\begin{pmatrix}
1&0\\-1&1
\end{pmatrix}$$
$$C^n=\begin{pmatrix}
1&0\\-n&1
\end{pmatrix}$$
$$C^n*C=\begin{pmatrix}
1&0\\-n&1
\end{pmatrix}\times\begin{pmatrix}
1&0\\-1&1
\end{pmatrix}=\begin{pmatrix}
1&0\\-(n+1)&1
\end{pmatrix}$$
$\therefore A\cdot B$ has infinite order.
\end{enumerate}

\end{document}